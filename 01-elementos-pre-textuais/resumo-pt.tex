%
% Documento: Resumo (Português)
%
\begin{resumo}

A gama de aplicação de quadricópteros tem crescido substancialmente nos últimos anos, sendo eles utilizados inclusive para fins militares. Entretanto, para que o seu uso seja possível, se faz necessário o desenvolvimento de controladores que permitam seu funcionamento adequado de forma a permitir sua estabilidade. Para tanto, foram desenvolvidos ao longo deste trabalho, para estabilizar a atitude e altitude de um \textit{drone}, dois controladores baseados em técnicas diferentes de Inteligência Computacional (IC): \textit{fuzzy} e neuro-\textit{fuzzy}. Os resultados obtidos por eles são comparados e mostra-se que ambos estabilizam o sistema de forma eficiente e, ainda, que com o controlador neuro-\textit{fuzzy} obtiveram-se melhores resultados, além de ele oferecer melhor desempenho energético principalmente sobre um sistema com ruído de medição. Com isto, mostra-se que técnicas de IC podem ser aplicadas no projeto e implementação de controladores eficientes para sistemas não-lineares complexos e que o poder de aprendizado das RNAs é, de fato, capaz de melhorar a performance de um sistema \textit{fuzzy}, o que pode ser constatado pelo melhor desempenho obtido pelo controlador neuro-\textit{fuzzy}.

%A gama de aplicação de helicópteros quadrotores tem crescido substancialmente nos últimos anos. Entretanto, para o uso adequado deles, faz-se necessário o desenvolvimento de controladores que permitam seu controle eficiente. Desta forma, este trabalho tem, como objetivo, o desenvolvimento de controladores Neuro-Fuzzy para controlar a atitude e altitude de um quadrotor, sendo que as aplicações da Inteligência Computacional vêm ganhando muito espaço recentemente, além do fato de, em diversos trabalhos na literatura, ser mostrado que controladores que utilizam técnicas de IC obtêm melhor desempenho do que controladores convencionais. Para tanto, em ambiente simulado, primeiramente contextualizou-se a necessidade de controladores para o controle de dois sistemas dinâmicos não lineares intrinsecamente instáveis: o de um quadrotor e também o de um pêndulo invertido, que é uma analogia comumente usada na literatura para sistemas dinâmicos não lineares devido aos aspectos similares de instabilidade aos observados nos quadrotores. Foram então projetados dois controladores Fuzzy para controlar a atitude e a altitude de um quadrotor quando submetido a diferentes perturbações. Então, a partir destes controladores Fuzzy, foram projetados dois controladores Neuro-Fuzzy com os mesmos propósitos. Por fim, mostrou-se que os Neuro-Fuzzy de fato melhoraram levemente a resposta do sistema diante do controle de duas das três variáveis controladas ao passo que piorou de forma considerável a sobrelevação no controle da terceira.

\textbf{Palavras-chave}: quadrotor. controle neuro-fuzzy. inteligência computacional.

\end{resumo}

%\iffalse
%\textbf{A ser escrito ao final do trabalho}. Síntese do trabalho em texto cursivo contendo um único parágrafo. O resumo é a apresentação clara, concisa e seletiva do trabalho.
%No resumo deve-se incluir, preferencialmente, nesta ordem: brevíssima introdução ao assunto do trabalho de pesquisa (qualificando-o quanto à sua natureza), o que será feito no trabalho (objetivos), como ele será desenvolvido (metodologia), quais serão os principais resultados e conclusões esperadas, bem como qual será o seu valor no contexto acadêmico. Para o projeto de dissertação sugere-se que o resumo contenha até 200 palavras.
%
%\textbf{Palavras-chave}: latex. abntex. modelo.
%(Entre 3 a 6 palavras ou termos, separados por ponto, descritores do trabalho. As palavras-chaves são Utilizadas para indexação.
%\fi


