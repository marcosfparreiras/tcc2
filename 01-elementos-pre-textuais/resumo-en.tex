%
% Documento: Resumo (Inglês)
%

\begin{resumo}[Abstract]

The range of applications of drones have grown substantially in the last years, as they have been used even for military services. However, in order to make their use possible, the development of controllers which allow their suitable operation to guarantee their stability is needed. Therefore, in this dissertation, two controllers based on Computational Intelligence (CI) techniques, fuzzy and neuro-fuzzy, were developed to stabilize a drone's attitude and altitude. The results that they obtained were compared and it's shown that both were able to stabilize the system with efficiency and yet that the neuro-fuzzy reached better results including a better energy efficiency even in the case in which a measurement noise was incorporated to the system. From it, one can see that CI techniques may be applied in the project and implementation of efficient controller for complex non linear systems and that the Neural Networks's power of learning is able to increase a fuzzy system's performance indeed, what can be seen from the better results obtained using the neuro-fuzzy controller.

\textbf{Keywords}: drone. neuro-fuzzy control. computational intelligence.

\end{resumo}


%However, in order to use them properly, it's needed the development of controllers that allow their suitable control. Thus, this work has, as main goal, the development of a Neuro-Fuzzy controller to control a quadcopter, since the Computational Intelligence applications has won space recently besides several works in literature have shown that controllers that use CI techniques, have improved performance then the conventional controllers. In order to do so, through simulations, first of all it was contextualized the requirement of a control system for two non-linear dynamic systems: a quadcopter and also an inverted pendulum, which is a commonly used analogy in literature for non-linear dynamic systems due to the similar istability aspects to the ones observed on the quadcopters. Two Fuzzy contollers were projected in order to control the quadcopter's attitude and altitude when subjected to different disturbances. Then, based on these Fuzzy controllers, two Neuro-Fuzzy controllers were designed for the same purposes. In the end, it was shown that the Neuro-Fuzzy slightly increased the control performance over two among the three controlled variables whereas it decreased substantially the performance over the third one, increasing considerably its overshoot.


