Como mostrado em \cite[p.~63]{Balas2007}, o sistema modelado, já incluindo o desacoplamento de entradas, pode ser representado da seguinte forma no espaço de estados.
\begin{align} \label{eq:space_state_equation_quadcopter}
	\dot{x} &= Ax + Bu \nonumber \\
	y		&= Cx + Du
\end{align}

Com o vetor de estados $X$ sendo dado por:
\begin{equation*}
X=
\left[ \begin{array}{@{}*{12}{c}@{}}
     x & y & z & \dot{x} & \dot{y} & \dot{z} & \phi & \theta & \psi & \dot{\phi} & \dot{\theta} & \dot{\psi}\\
\end{array} \right]^T
\end{equation*}

O vetor de entrada $U$ sendo:
\[ 
	U =
	\begin{bmatrix}
		u_1 & 
		u_{2\textunderscore decoupled} &
		u_{3\textunderscore decoupled} &
		u_{4\textunderscore decoupled}
	\end{bmatrix}^T
\]

E as matriz A, B, C e D sendo definida como:
\begin{equation*}
A =
\left[ \begin{array}{@{}*{12}{c}@{}}
     0 & 0 & 0 & 1 & 0 & 0 & 0 & 0  & 0 & 0 & 0 & 0 \\
     0 & 0 & 0 & 0 & 1 & 0 & 0 & 0  & 0 & 0 & 0 & 0 \\
     0 & 0 & 0 & 0 & 0 & 1 & 0 & 0  & 0 & 0 & 0 & 0 \\
     0 & 0 & 0 & 0 & 0 & 0 & 0 & -g & 0 & 0 & 0 & 0 \\
     0 & 0 & 0 & 0 & 0 & 0 & g & 0  & 0 & 0 & 0 & 0 \\
     0 & 0 & 0 & 0 & 0 & 0 & 0 & 0  & 0 & 0 & 0 & 0 \\
     0 & 0 & 0 & 0 & 0 & 0 & 0 & 0  & 0 & 1 & 0 & 0 \\
     0 & 0 & 0 & 0 & 0 & 0 & 0 & 0  & 0 & 0 & 1 & 0 \\
     0 & 0 & 0 & 0 & 0 & 0 & 0 & 0  & 0 & 0 & 0 & 1 \\
     0 & 0 & 0 & 0 & 0 & 0 & 0 & 0  & 0 & 0 & 0 & 0 \\
     0 & 0 & 0 & 0 & 0 & 0 & 0 & 0  & 0 & 0 & 0 & 0 \\
     0 & 0 & 0 & 0 & 0 & 0 & 0 & 0  & 0 & 0 & 0 & 0 \\
\end{array}\right]
\end{equation*}

\begin{equation*}
B =
\left[\begin{array}{@{}*{4}{c}@{}}
	0 & 0 & 0 & 0 \\
	0 & 0 & 0 & 0 \\
	0 & 0 & 0 & 0 \\
	0 & 0 & 0 & 0 \\
	0 & 0 & 0 & 0 \\
	-1/m & 0 & 0 & 0 \\
	0 & 0 & 0 & 0 \\
	0 & 0 & 0 & 0 \\
	0 & 0 & 0 & 0 \\
	0 & cos(\psi)/I_{xx} & -sen(\psi)/I_{yy} & 0 \\
	0 & sen(\psi)/(cos(\phi)I_{xx}) & cos(\psi)/(cos(\phi)I_{yy}) & 0\\
	0 & sin(\psi)tan(\phi)/I_{xx} & cos(\psi)*tan(\phi)/I_{yy} & 1/I_{zz} \\
\end{array}\right]
\end{equation*}

\begin{equation*}
C =
\left[ \begin{array}{@{}*{12}{c}@{}}
	1 & 0 & 0 & 0 & 0 & 0 & 0 & 0 & 0 & 0 & 0 & 0 \\
	0 & 1 & 0 & 0 & 0 & 0 & 0 & 0 & 0 & 0 & 0 & 0 \\
	0 & 0 & 1 & 0 & 0 & 0 & 0 & 0 & 0 & 0 & 0 & 0 \\
	0 & 0 & 0 & 1 & 0 & 0 & 0 & 0 & 0 & 0 & 0 & 0 \\
	0 & 0 & 0 & 0 & 1 & 0 & 0 & 0 & 0 & 0 & 0 & 0 \\
	0 & 0 & 0 & 0 & 0 & 1 & 0 & 0 & 0 & 0 & 0 & 0 \\
	0 & 0 & 0 & 0 & 0 & 0 & 1 & 0 & 0 & 0 & 0 & 0 \\
	0 & 0 & 0 & 0 & 0 & 0 & 0 & 1 & 0 & 0 & 0 & 0 \\
	0 & 0 & 0 & 0 & 0 & 0 & 0 & 0 & 1 & 0 & 0 & 0 \\
	0 & 0 & 0 & 0 & 0 & 0 & 0 & 0 & 0 & 1 & 0 & 0 \\
	0 & 0 & 0 & 0 & 0 & 0 & 0 & 0 & 0 & 0 & 1 & 0 \\
	0 & 0 & 0 & 0 & 0 & 0 & 0 & 0 & 0 & 0 & 0 & 1 \\
\end{array}\right]
\end{equation*}

\begin{equation*}
D =
\left[\begin{array}{@{}*{4}{c}@{}}
	0 & 0 & 0 & 0 \\
	0 & 0 & 0 & 0 \\
	0 & 0 & 0 & 0 \\
	0 & 0 & 0 & 0 \\
	0 & 0 & 0 & 0 \\
	0 & 0 & 0 & 0 \\
	0 & 0 & 0 & 0 \\
	0 & 0 & 0 & 0 \\
	0 & 0 & 0 & 0 \\
	0 & 0 & 0 & 0 \\
	0 & 0 & 0 & 0 \\
	0 & 0 & 0 & 0 \\
\end{array}\right]
\end{equation*}
em que $g$ é a gravidade, $m$, a massa do quadrotor e os ângulos $\phi$, $\theta$ e $\psi$ representam os ângulos em torno dos eixos $x$, $y$ e $z$ respectivamente. Além disto, $I_{xx}$, $I_{yy}$ e $I_{zz}$ representam o momento de inércia do quadricóptero ao longo destes mesmos eixos.

Ao longo deste trabalho, essa foi a modelagem adotada para simular o sistema de um quadrotor.