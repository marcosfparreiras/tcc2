%
% Documento: Conclusão
%

\chapter{Conclusão}
\label{chap:conclusao}

Ao longo deste trabalho, discorreu-se sobre o crescente uso de estratégias da Inteligência Computacional para implementar controladores de sistemas não lineares. Além disto, como foi mostrado pelos experimentos computacionais realizados, o uso de controladores devidamente projetados é fundamental para fazer com que esses sistemas instáveis atuem da forma planejada e possam ser estabilizados.

No contexto deste trabalho, o sistema controlado é um quadricóptero e as variáveis são referentes à sua atitude e altitude, representando portanto um controle multivariável. As alternativas propostas como controladores foram o \textit{fuzzy} e o neuro-\textit{fuzzy}. A opção pelo primeiro se deveu ao fato de ele permitir a modelagem a partir de variáveis e termos linguísticos, além de acrescer robustez ao sistema. Já a opção pelo segundo, neuro-fuzzy, foi devido ao fato de este agregar as características de RNAs aos sistemas fuzzy, possuindo um poder de aprendizado capaz de melhorar sua performance.

De fato, os resultados mostram que o controlador neuro-\textit{fuzzy} realmente obteve melhor desempenho. No controle de altitude, reduziu o tempo de convergência em 29\% e a variação do sistema em 31\%, além de eliminar a sobrelevação apresentada pelo \textit{fuzzy}. O controle neuro-\textit{fuzzy} de atitude também apresentou melhoras, apesar de não tão significativas quanto essas: reduziu o tempo de convergência em 2\% e a variação do sistema em 13\%.

Além disto, num teste para verificar a robustez dos controladores, a massa do sistema foi acrescida em 117\%, passando de 2 kg para 5 kg. Neste novo contexto, o controlador de atitude neuro-\textit{fuzzy} mais uma vez foi superior ao \textit{fuzzy}, reduzindo o tempo de convergência em 3\% e da variação em 14\%. Já no controle de altitude, o único fator melhorado pelo neuro-\textit{fuzzy} foi a variação do sistema, sendo reduzida em 20\%, ao passo que seu tempo de convergência cresceu 57\%, além de ter sido inserida uma sobrelevação na resposta.

No experimento computacional incluindo ruídos de medição, ambos os controladores se mostraram capazes de lidar com eles. Neste ponto, o controlador neuro-\textit{fuzzy} apresentou desempenho bastante superior, levando a uma convergência mais rápida e com menor variação além de consumir menos energia.

Apesar das diferenças de desempenho, os controladores \textit{fuzzy} e neuro-\textit{fuzzy} tanto para atitude quanto para altitude do sistema foram capazes de estabilizá-lo, mesmo quando submetido a uma variação substancial de parâmetros, que foi representada pelo aumento da massa em mais de 100\%.

Desta forma, mostra-se que se podem usar técnicas de IC para controlar, de forma eficiente, sistemas não lineares complexos e, além disso, que controladores neuro-\textit{fuzzy} podem ser utilizados para melhorar o desempenho de controladores \textit{fuzzy} apesar de, em algumas situações, piorar a resposta se comparado a estes.

\section{Trabalhos futuros}
\label{sec:conclusao-trabalhosFuturos}

Os resultados obtidos nesta dissertação abrem espaço para diferentes frentes de trabalho, tais como:
\begin{itemize}
\item Implementação dos controladores propostos sobre um sistema físico ;
\item Investigação mais completa sobre a resposta dos controladores desenvolvidos quando atuando sobre sistemas sujeitos a diferentes tipos de ruído, tais como de medição e de atuação;
\item Comparar os resultados obtidos pelos controladores desenvolvidos aos obtidos por outros controladores que implementam técnicas de IC e  outros que seguem técnicas tradicionais de controle.
\end{itemize}


