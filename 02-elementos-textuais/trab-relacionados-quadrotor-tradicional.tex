Esta seção aborda diferentes propostas de controladores tradicionais para quadricópteros visando a contextualização do estado da arte para depois se poder comparar alguns resultados das técnicas tradicionais e inteligentes de controle para esse sistema.

Em \cite{Razinkova2014}, foi proposto um controlador PD para a descrição adequada das trajetórias definidas de um quadricóptero \textit{Indoor}. Para permitir um funcionamento adequado \textit{Outdoor}, foi integrado ao controlador PD um controle adaptativo, adicionando termos ao controlador convencional, de forma a permitir o ajuste correto da trajetória do quadricóptero quando submetido a distúrbios nos eixos X e Y (plano horizontal). Os resultados mostraram que o controlador adaptativo melhorou consideravelmente o controle de trajetória do \textit{drone}, reduzindo em 64\% o erro de posicionamento para os casos de trajetória em linha reta ao longo do plano XY, e em 72\% para os casos de trajetória circular sobre o plano XY. Segundo os autores, este resultado representa uma melhora significativa, uma vez que o controlador resultante possui uma arquitetura simples e não requer computação extensiva, o que é indesejado para qualquer controle de sistema, especialmente para UAVs, devido à sua limitação de bateria.

Em \cite{Mustapa2014} é proposto um controlador PID para efetuar o controle sobre a estabilidade da altitude de um helicóptero quadrotor. Neste trabalho, foi utilizado um modelo matemático desenvolvido previamente para descrever o comportamento do sistema. Para ajustar os parâmetros do controlador PID, foi necessário realizar um experimento para determinar o momento de inércia do drone. Uma vez levantados os valores necessários, um controlador PID foi ajustado e se mostrou eficiente. A simulação do sistema se deu no Mat-lab Simulink.

\citeonline{Khatoon2014} também propuseram um controlador PID para controlar a estabilidade em altitude de um drone, mantendo a posição no eixo XY constante, mesmo esta altitude sendo uma variável muito sensível a mudanças em outros parâmetros. No trabalho, é mostrado que um controlador PID sozinho é capaz de exercer tal controle com robustez. A escolha deste controlador se deu graças à sua robustez e facilidade de modelagem. Entretanto, é também apontado pelos autores que, apesar da simplicidade para se modelar um controlador PID, isto requer uma modelagem do sistema como um todo, o que não é fácil, devido à sua estrutura complexa, suas dinâmicas não lineares e sua natureza subatuada. O sistema modelado para o drone mostrou ser altamente instável, justificando a necessidade de um controlador. A partir de extensivas simulações no MATLAB/Simulink, o sistema desenvolvido se mostrou bem sucedido, implementando de fato um controle robusto de altitude para um helicóptero quadrotor.

Além desses controladores que aplicam técnicas tradicionais, são também vários os exemplos de trabalhos que retratam o controle inteligente de quadrotores.