\chapter{Metodologia}
\label{chap:metodologia}

Todo o trabalho foi desenvolvido em ambiente simulado, utilizando o \textit{software} Matlab\textsuperscript{\textregistered}. Primeiramente, foram representados no Simulink\textsuperscript{\textregistered} dois sistemas de quadricóptero seguindo a modelagem proposta por \citeonline{Balas2007}: ambos submetidos a uma gravidade $g = 9,8$ m/s\textsuperscript{2} e com comprimento de cada haste $l = 0,5$ m. Os dois sistemas diferem entretanto na massa do quadricóptero modelado em cada caso: $m = 2,3$ kg e $m = 5$ kg.

O sistema com massa $m = 2,3$ kg foi então utilizado para mostrar o desacoplamento das entradas e a instabilidade do sistema. Para tanto, o modelo foi submetido a sinais em pulso em cada uma de suas entradas. Então, a partir da resposta do sistema a essas entradas, foram modelados dois controladores \textit{fuzzy} para estabilizar a atitude e altitude do quadricóptero. Para tanto, foi utilizada a ferramenta \textit{Fuzzy Logic Toolbox} do Matlab\textsuperscript{\textregistered}.

A partir dos controladores \textit{fuzzy} desenvolvidos e utilizando a ferramenta \textit{Neuro-Fuzzy Designer} também do Matlab\textsuperscript{\textregistered} foram modelados dois controladores neuro-\textit{fuzzy} para controlar a atitude e altitude do \textit{drone}.

Os controladores \textit{fuzzy} e neuro-\textit{fuzzy} foram então comparados tanto para o sistema com massa de $m = 2,3$ kg quanto para o de $m = 5$ kg. Os aspectos levados em conta para a comparação dos controladores foram:
\begin{itemize}
	\item Variação apresentada;
	\item Tempo necessário para a estabilização;
	\item Oscilação;
	\item Sobrelevação apresentada;
	\item Gasto enérgico apresentado pelos controladores.
\end{itemize}

Por fim, o sistema com massa $m$ = 2,3 kg, para o qual os controladores foram desenvolvidos, foi submetido a um cenário que envolve ruídos de medição para verificar se o controle implementado se mostra robusto.
