Como se pode ver, os controladores de atitude e altitude tanto fuzzy quanto neuro-fuzzy foram eficientes levando à estabilização do sistema em todos os casos testados, inclusive na situação em que a massa do sistema foi aumentada em mais de 100 \%. Isso indica que estes controladores podem ser utilizados, por exemplo, em situações em que o drone precisaria transportar uma carga que tenha sua massa ou até mesmo uma superior.

Em quase todos os casos, nota-se um comportamento do controlador neuro-fuzzy superior ao do fuzzy, o que já era esperado tendo em vista que o primeiro alia o poder do segundo às vantagens das RNAs, fazendo com que, a partir de um treinamento supervisionado utilizando o próprio modelo fuzzy, possa se construir um controle mais abrangente e com resposta melhorada.

Estes resultados mostram que, de fato, as técnicas de Inteligência Computacional podem ser aplicadas para projetar controladores eficientes e robustos para atuar sobre sistemas multivariável e que, além disto, o poder de treinamento dos ANFISs realmente é capaz de fazer com que o desempenho de controladores neuro-fuzzy seja melhorado se comparado a de controladores puramente fuzzy.
