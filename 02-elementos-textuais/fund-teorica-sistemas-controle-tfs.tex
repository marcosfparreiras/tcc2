%Ogata 15 do pdf; Dorf 65 do pdf
A função de transferência de um sistema representa a relação que descreve as dinâmicas do sistema em questão e é definida pela razão entre as transformadas de Laplace das variáveis de saída e de entrada com todas as condições iniciadas definidas como zero \cite[p.~65]{Dorf2011}. A forma de uma função de transferência é dada a seguir:
\begin{center}
$G(s) = \frac{Y(s)}{X(s)}$
\end{center}
em que $G(s)$ é a função de transferência que descreve o sistema, $Y(s)$ é a transformada de Laplace da variável de saída do sistema e $X(s)$ é a transformada de Laplace da variável de entrada, ambas considerando as condições iniciais definidas como zero.

Apesar de ser amplamente utilizada, o uso de funções de transferência tem certas limitações. As transformadas de Laplace só podem ser utilizadas sobre sistemas descritos por equações e diferencias lineares e sem parâmetros variantes no tempo e, portanto, o uso das funções de transferência se restringem aos casos em que estas condições são satisfeitas.

Além da representação a partir de funções de transferência, uma forma alternativa de representar as dinâmicas do sistema é a representação no espaço de estados.

