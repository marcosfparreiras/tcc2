\begin{quadro}[!htb]
    \centering
    \caption{Hierarquia de restrições das questões\label{qua:comparabd}}
    \begin{tabular}{|p{7cm}|p{7cm}|}
        \hline
        \textbf{BD Relacionais} & \textbf{BD Orientados a Objetos} \\
        \hline
        Os dados são passivos, ou seja, certas operações limitadas podem ser automaticamente acionadas quando os dados são usados. Os dados são ativos, ou seja, as solicitações fazem com que os objetos executem seus métodos. & Os processos que usam dados mudam constantemente. \\
        \hline
    \end{tabular}
    \fonte{\citeonline{carvalho:2001}}
\end{quadro}
