\begin{lstlisting}[inputencoding=latin1]
	% le arquivo fis referente ao controle de altitude
	fismat = readfis('fis_altitude.fis');
	
	% define numero de casos a serem avaliados (treinamento + teste)
	n = 300;
	% define conjunto de n entradas aleatorias para o sistema fuzzy
	% respeitando o range de cada entrada
	input = zeros(n,2);
	for i=1:n    
	    z_value = rand * 2 - 1;
	    z_dot_value = rand * 10 - 5;
	    input(i,:) = [ z_value z_dot_value ];
	end
	
	% avalia resposta fuzzy para cada entrada
	output= evalfis(input,fismat);
	
	% define data como vetor relacionando cada conjunto de entradas a saida
	% - obtida pelo sistema fuzzy
	data = [];
	for i=1:n
	   data(i,:) = [ input(i,:) output(i) ];
	end
	
	% define que 2/3 dos dados obtidos serao usasdos para treinamento
	% e 1/3 sera usado para teste da rede
	train = data(1:2*n/3,:);    % dados para treinamento
	test = data(2*n/3+1:n,:);   % dados para validacao do sistema treinado
	
	% gera modelo fuzzy Sugeno a partir do Mamdani modelado
	sugFIS = mam2sug(fismat);
	% salva modelo Sugeno em disco com o nome fis_altitude_neuro.fis
	writefis(sugFIS, 'fis_altitude_neuro.fis');
\end{lstlisting}
